\chapter{Introduzione}

\paragraph{}
(prima bozza, verrà raffinata al termine della stesura dei capitoli)

\paragraph{}
E' sempre più evidente che il cloud computing è il futuro del software. La rivoluzione consta nella distribuzione dei servizi di calcolo e nella virtualizzazione delle risorse, dando così all'utente la sensazione di un utilizzo centralizzato. Tutto ciò si è reso realizzabile dal momento in cui l'accesso alla rete è divenuto possibile da sempre più dispositivi e con velocità di connessione sempre maggiori.
\paragraph{}
La tematica del cloud computing è stata centrale nel mio lavoro di tesi presso l'azienda IBM (International Business Machines Corporation) nella sua sede di Roma. Ho partecipato attivamente alla realizzazione di un prototipo software, ossia la versione SaaS della suite IBM BigFix. BigFix è una suite di prodotti dedicati alle aziende che risolvono problematiche di Endpoint Security e di compliance di dispositivi a determinate politiche aziendali. Tramite questi prodotti si ottiene pieno controllo su tutti i dispositivi aziendali. Si possono ad esempio rilevare eventuali attacchi o si possono distribuire aggiornamenti e patch.
\paragraph{}
La sfida da me raccolta è quindi proprio quella di portare tutto questo arsenale di strumenti nella leggerezza del cloud. Rendendolo disponibile, nel giro di pochi minuti, anche a chi è sempre stato intimorito dalla difficoltà di installazione di uno strumento così potente, ma allo stesso tempo complesso.