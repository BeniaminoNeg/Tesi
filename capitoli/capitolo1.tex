\chapter{Introduzione}

\paragraph{}
E' sempre più evidente che il cloud computing è il futuro del software. Il suo avvento rappresenta una rivoluzione nella fruizione dell'informazione.
\paragraph{}
Il profondo cambiamento consta nella distribuzione dei servizi di calcolo e nella virtualizzazione delle risorse. Gli utenti hanno così la sensazione di un utilizzo centralizzato. Tutto ciò è divenuto possibile dal momento in cui l'accesso alla rete avviene da sempre più dispositivi e con velocità di connessione sempre maggiori.
\paragraph{}
La tematica del cloud computing è stata centrale nel lavoro di tesi svolto presso l'azienda IBM (International Business Machines Corporation) nella sua sede di Roma. Si è realizzato nello specifico un prototipo software: la versione SaaS dei prodotti IBM BigFix. BigFix è una suite di prodotti dedicati alle aziende che risolvono problematiche di Endpoint Security e di compliance di dispositivi a politiche aziendali ottenendo il pieno controllo su tutti i dispositivi.
\paragraph{}
La sfida raccolta è quella di portare l'arsenale di strumenti BigFix nella leggerezza del cloud. Rendendolo disponibile, nel giro di pochi minuti, anche ai clienti che non potevano permettersi la versione on premise. Anche a chi è sempre stato intimorito dalla difficoltà che comportava l'installazione di uno strumento così potente, ma allo stesso tempo complesso.

\paragraph{La diffusione del Cloud Computing}
L'utilizzo di servizi cloud è ormai sempre più diffuso. A volte questo è consapevole altre meno. Si è molto più consci del suo utilizzo ad esempio quando si utilizzano piattaforme nelle quali è possibile archiviare i propri dati personali, come Dropbox o Google Drive. Ma in realtà il concetto di cloud è molto più ampio. 
\paragraph{}
Dividiamo gli strumenti cloud in tre grandi macro categorie: Software as a Service (SaaS), Platform as a Service (PaaS) e Infrastructure as a Service (IaaS). Mentre le IaaS mettono a disposizione attraverso la rete infrastrutture e le PaaS offrono dei componenti software da assemblare, un SaaS è un vero è proprio prodotto software completo, che l'utente può utilizzare da qualunque dispositivo voglia.
\paragraph{}
Rientra proprio in questa categoria il sistema che è stato realizzato in questo lavoro di tesi. Una versione SaaS dei prodotti BigFix di IBM è un'importante scelta strategica aziendale perchè porta il servizio a essere adatto a un panorama molto più vasto di clienti.