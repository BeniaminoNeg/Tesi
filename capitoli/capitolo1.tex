\chapter{Introduzione}

\paragraph{}
E' sempre più evidente che il cloud computing è il futuro del software. Il suo avvento rappresenta una vera rivoluzione nella fruizione dell'informazione.
\paragraph{}
Il profondo cambiamento consta nella distribuzione dei servizi di calcolo e nella virtualizzazione delle risorse in modo da conferire agli utenti la sensazione di un utilizzo centralizzato. Tutto ciò è divenuto realizzabile dal momento in cui l'accesso alla rete è divenuto possibile da sempre più dispositivi e con velocità di connessione sempre maggiori.
\paragraph{}
La tematica del cloud computing è stata centrale nel lavoro di tesi svolto presso l'azienda IBM (International Business Machines Corporation) nella sua sede di Roma. Si è realizzato nello specifico un prototipo software: la versione SaaS dei prodotti IBM BigFix. BigFix è una suite di strumenti dedicati alle aziende che risolvono problematiche di Endpoint Security e di compliance di dispositivi a politiche aziendali. Con il suo utilizzo si ottiene il pieno controllo su tutti gli endpoint.
\paragraph{}
La sfida raccolta è quella di portare l'arsenale di strumenti BigFix nella leggerezza del cloud. Rendendolo disponibile, nel giro di pochi minuti, anche ai clienti che non potevano permettersi l'adozione della versione tradizionale per ragioni tecniche. Ossia anche a coloro che erano sempre stati intimoriti dai requisiti infrastrutturali che comportava l'installazione di uno strumento così potente, ma allo stesso tempo complesso.

\paragraph{La diffusione del Cloud Computing}
L'utilizzo di servizi cloud è ormai sempre più comune. A volte in maniera consapevole, altre meno. Si è molto più consci del suo utilizzo, ad esempio, quando si usufruisce di piattaforme per l'archiviazione dei propri dati personali, come Dropbox o Google Drive. Ma in realtà il concetto di cloud è molto più ampio. Esso si estende a tutte le tipologie di servizi che si attengono a una architettura cloud. Possiamo utilizzare un software in cloud ad esempio per gestire la posta elettronica, o utilizzarlo per la gestione delle vendite di un magazzino, fino anche alla gestione di alcuni aspetti aziendali, come nel caso di questo lavoro di tesi.
\paragraph{}
I vantaggi dati dall'utilizzo di un servizio cloud sono molteplici. Sicuramente l'aspetto più apprezzabile è la possibilità di accedere al servizio stesso da qualunque dispositivo disponga una connessione ad internet, senza bisogno di installazioni. Un altro aspetto fondamentale, soprattutto per i prodotti professionali, è la facilità di installazione. I servizi cloud sollevano l'utente dalla necessità di possedere l'hardware necessario a far funzionare il sistema, basta un semplice strumento come un browser web. Infine è bene ricordare che, essendo gestiti completamente dal provider, questi prodotti possono avere prestazioni molto maggiori dei software tradizionali, chiamati on premise.
\paragraph{}
Dividiamo gli strumenti cloud in tre grandi macro categorie: Software as a Service (SaaS), Platform as a Service (PaaS) e Infrastructure as a Service (IaaS). Mentre le IaaS mettono a disposizione attraverso la rete le infrastrutture e le PaaS offrono dei componenti software da assemblare, un SaaS è un vero è proprio prodotto software completo, che l'utente può utilizzare da qualunque dispositivo si desideri.

\paragraph{}
Rientra proprio in quest'ultima categoria il sistema che è stato realizzato nel lavoro di tesi. Una versione SaaS dei prodotti BigFix di IBM è un'importante scelta strategica aziendale perché porta il servizio a essere appetibile a un panorama molto più vasto di clienti, data la facilità di adozione per le aziende di una tecnologia cloud rispetto a una on premise.
\paragraph{}
Portare BigFix su SaaS è il progetto sperimentale più sfidante della divisione Security di IBM. Esso è un prodotto oramai affermato, ma c'è la necessità di non rimanere solamente ancorati alla versione standard del prodotto. Per questo mi è stato affiancato nel lavoro un team molto valido di altri due developer guidati da un architect in IBM con lo scopo di progettare e implementare il prototipo SaaS. Questo verrà poi attentamente valutato prima di essere lanciato definitivamente sul mercato.

\paragraph{Il problema da risolvere}
Passare un prodotto on-premise, con la sua architettura molto articolata, su cloud non è un problema di semplice risoluzione. Occorre rivoluzionarne completamente la struttura, adottare nuove tecnologie, modificare le componenti esistenti e creare nuovi moduli software. 

\paragraph{}
La differenza strutturale principale sta nel fatto che, in una soluzione SaaS, tutte le componenti software, le risorse hardware e i dati di tutti i clienti risiedono presso il provider, in questo caso IBM stessa. Il cliente possiede presso la sua sede solamente l'interfaccia con la quale accedere a tutte queste risorse e questo è un grande vantaggio per il cliente stesso, ma anche un importante problema da risolvere. 

\paragraph{L'approccio alla soluzione}
Per meglio progettare il prototipo che si va a realizzare è necessaria un'attenta fase di identificazione dei requisiti, sia funzionali che qualitativi. A questo proposito si sono adottati dei framework agile per condurre l'intero lavoro. 
\paragraph{}
La fase di ideazione del progetto è avvenuta dopo un attenta fase di "Design Thinking", un processo creativo ormai consolidato in IBM che coinvolge fin dall'inizio progettisti, sviluppatori e clienti. In questo modo si cerca di instaurare un rapporto di empatia tra tutti gli stackeholders del progetto e comprendere al meglio le necessità degli utenti finali.
\paragraph{}
Il processo di sviluppo è iterativo e incrementale. E' fondamentale infatti realizzare da subito componenti del sistema eseguibili e mostrabili agli utenti, i microservizi, in modo da massimizzare le opportunità di feedback. Si è adottata nel team la metodologia di sviluppo agile SCRUM, un processo che possiamo definire "empirista" in quanto segue il principio che la conoscenza deriva solamente dall'esperienza. Occorre avere cicli di produzione software molto brevi, nei quali si deve ciclicamente fare analisi, progettazione, implementazione e testing in modo da cogliere sempre più pareri dagli stackeholders. Si ha sempre più coscienza così del prodotto che si deve realizzare, producendo un servizio che risponde esattamente alle esigenze che lo hanno richiesto.

\paragraph{La soluzione}
Con questa filosofia si definisce l'intera architettura di BigFix SaaS. Si è scelto di scomporre l'intero sistema in microservizi, ossia piccole entità software completamente indipendenti tra loro. Ognuna di queste esegue dei task atomici necessari per soddisfare le esigenze degli utenti. Vengono poi opportunamente assemblati, replicati e distribuiti nei data-center per garantire la maggiore efficienza possibile. La comunicazione tra essi avviene attraverso la rete.
\paragraph{}
Per realizzare un'architettura a microservizi occorre integrare alcuni tool nati proprio per il panorama cloud. Tra questi spiccano Docker e Kubernetes che consentono di gestire particolari strutture software chiamate container. Come suggerisce il termine stesso, il loro scopo è quello di contenere e isolare i microservizi per renderli facilmente replicabili e indipendenti dalle tecnologie sottostanti.
\paragraph{}
Un altro fattore fondamentale è stata l'adozione di pratiche di DevOps e continuous delivery. Queste fanno fronte a necessità particolari portate dalla filosofia cloud. Il loro scopo è fare in modo che si automatizzino tutti i processi di gestione, come l'adozione del prodotto da parte di un nuovo cliente.

\paragraph{Per orientarsi nella consultazione di questa tesi}
Per orientarsi nella consultazione di questo elaborato andiamo a spiegare brevemente la sua struttura e le principali tematiche di trattazione. Nel capitolo 2 verrà descritto il contesto in cui si è svolta questa tesi in azienda. Verrà presentato il prodotto di IBM che si è portato su SaaS e verrà fatta una panoramica delle metodologie di lavoro.
\paragraph{}
Nei capitoli 3 e 4 ci concentreremo sulla tecnologia del cloud computing e sulle sue diverse sfaccettature in cui si trova implementato. Un cambiamento importante come quello cloud comporta l'impiego di tecnologie apposite che verranno presentate poi nel capitolo 5.
\paragraph{}
Giungiamo così finalmente al capitolo 6 dove si espone in maniera molto approfondita il percorso progettuale che ha portato alla definizione dell'architettura del SaaS di BigFix. Qui si vedrà come l'attuazione del Design Thinking e del framework SCRUM ha portato a definire l'Architettura del prodotto che vede la sua implementazione descritta, passo dopo passo, nel capitolo 7.