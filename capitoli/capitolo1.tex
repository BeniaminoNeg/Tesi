\chapter{Introduzione}

\paragraph{}
E' sempre più evidente che il cloud computing è il futuro del software. Il suo avvento rappresenta una rivoluzione nella fruizione dell'informazione.
\paragraph{}
Il profondo cambiamento consta nella distribuzione dei servizi di calcolo e nella virtualizzazione delle risorse. Gli utenti hanno così la sensazione di un utilizzo centralizzato. Tutto ciò è divenuto possibile dal momento in cui l'accesso alla rete avviene da sempre più dispositivi e con velocità di connessione sempre maggiori.
\paragraph{}
La tematica del cloud computing è stata centrale nel lavoro di tesi svolto presso l'azienda IBM (International Business Machines Corporation) nella sua sede di Roma. Si è realizzato nello specifico un prototipo software: la versione SaaS dei prodotti IBM BigFix. BigFix è una suite di prodotti dedicati alle aziende che risolvono problematiche di Endpoint Security e di compliance di dispositivi a politiche aziendali ottenendo il pieno controllo su tutti i dispositivi.
\paragraph{}
La sfida raccolta è quella di portare l'arsenale di strumenti BigFix nella leggerezza del cloud. Rendendolo disponibile, nel giro di pochi minuti, anche ai clienti che non potevano permettersi la versione on premise. Anche a chi è sempre stato intimorito dalla difficoltà che comportava l'installazione di uno strumento così potente, ma allo stesso tempo complesso.

\paragraph{La diffusione del Cloud Computing}
L'utilizzo di servizi cloud è ormai sempre più diffuso. A volte in maniera consapevole altre meno. Si è molto più consci del suo utilizzo ad esempio quando si utilizzano piattaforme nelle quali è possibile archiviare i propri dati personali, come Dropbox o Google Drive. Ma in realtà il concetto di cloud è molto più ampio. Esso si estende a tutte le tipologie di servizi che si attendono al paradigma cloud. Possiamo utilizzare un software in cloud per gestire la posta elettronica, o utilizzarlo per la gestione delle vendite di un magazzino fino anche alla gestione di alcuni aspetti aziendali, come nel caso di questo lavoro di tesi.
\paragraph{}
I vantaggi dati dall'utilizzo di un servizio cloud sono molteplici. Sicuramente l'aspetto più apprezzabile è la possibilità di accedere al servizio da qualunque dispositivo si abbia una connessione ad internet. Un'altro aspetto fondamentale, soprattutto per i prodotti professionali, è la facilità di installazione. I servizi cloud sollevano l'utente dalla necessità di possedere l'hardware necessario a far funzionare il sistema, basta un semplice strumento come un browser web. Infine è bene ricordare che, essendo gestiti completamente dal provider, questi prodotti possono avere prestazioni molto maggiori dei software tradizionali, chiamati on premise.
\paragraph{}
Dividiamo gli strumenti cloud in tre grandi macro categorie: Software as a Service (SaaS), Platform as a Service (PaaS) e Infrastructure as a Service (IaaS). Mentre le IaaS mettono a disposizione attraverso la rete infrastrutture e le PaaS offrono dei componenti software da assemblare, un SaaS è un vero è proprio prodotto software completo, che l'utente può utilizzare da qualunque dispositivo voglia.

\paragraph{}
Rientra proprio in questa categoria il sistema che è stato realizzato in questo lavoro di tesi. Una versione SaaS dei prodotti BigFix di IBM è un'importante scelta strategica aziendale perchè porta il servizio a essere appetibile a un panorama molto più vasto di clienti, data la facilità di adozione per le aziende di una tecnologia cloud rispetto a una on premise.
\paragraph{}
Portare Bigix su SaaS è il progetto più sfidante della divisione Security di IBM. BigFix è un prodotto ormai affermato, ma c'è la necessità di non rimanere solamente ancorati alla versione standard del prodotto. Per questo è stato affiancato un team molto valido di altri due developer guidati da un architect in IBM con lo scopo di progettare e implementare il prototipo SaaS che verrà poi attentamente valutato prima di essere messo definitivamente sul mercato.

\paragraph{Il problema da risolvere}
Passare un prodotto on-premise, con la sua architettura molto articolata, su cloud non è un problema di semplice risoluzione. Occorre rivoluzionarne completamente la struttura, adottare nuove tecnologie, modificare le componenti esistenti e creare nuovi moduli software. 

\paragraph{}
La differenza strutturale principale sta nel fatto che, in realtà, tutte le componenti software, le risorse hardware e i dati di tutti i clienti risiedono presso il provider, in questo caso IBM stessa. Il cliente possiede presso la sua sede solamente l'interfaccia con la quale accedere a tutte queste risorse e questo è un grande vantaggio per il cliente stesso, ma anche un importante problema da risolvere. 

\paragraph{L'approccio alla soluzione}
Per meglio progettare il prototipo che si realizza è necessaria un'attenta fase di identificazione dei requisiti sia funzionali che qualitativi. A questo proposito si sono abbracciati dei framework agile per condurre l'intero lavoro. 
\paragraph{}
La fase di ideazione del progetto è avvenuta dopo un attenta fase di "Design Thinking", un processo creativo ormai consolidato in IBM che coinvolge fin dall'inizio progettisti, sviluppatori, clienti e team di operations. In questo modo si cerca di instaurare un rapporto di empatia tra tutti gli stackeholders del progetto e comprendere al meglio le necessità degli utenti finali.
\paragraph{}
Il processo di sviluppo è iterativo e incrementale. E' fondamentale infatti realizzare da subito componenti del sistema eseguibili e mostrabili in modo da massimizzare le opportunità di feedback. Si è adottata nel team la metodologia di sviluppo SCRUM, un processo che possiamo definire "empirista" in quanto segue il principio che la conoscenza deriva solamente dall'esperienza. Occorre quindi avere cicli di produzione software molto brevi, nei quali si deve ciclicamente fare analisi, progettazione, implementazione, testing e cogliere sempre più feedback dagli stackeholders. Si accresce in questo modo sempre più la conoscenza del prodotto che si deve realizzare realizzando così un servizio che risponde esattamente alle esigenze che lo hanno richiesto.

\paragraph{La soluzione}
Si definisce così l'intera architettura di BigFix SaaS. Si è scelto di scomporre l'intero sistema in microservizi, ossia piccole entità software completamente indipendenti tra loro. Ognuna di queste esegue dei task atomici necessari per soddisfare le esigenze degli utenti. Vengono poi opportunamente assemblati, replicati e distribuiti nei data center per garantire la maggiore efficienza possibile. La comunicazione tra essi avviene attraverso la rete.
\paragraph{}
Per realizzare un'architettura a microservizi di questo tipo occorre integrare alcuni tool nati proprio per il panorama cloud. Tra questi spiccano Docker e Kubernetes che consentono di gestire particolari strutture software chiamate container. Come suggerisce il termine il loro scopo è quello di contenere e isolare i microservizi per renderli facilmente replicabili e indipendenti dalle tecnologie sottostanti.
\paragraph{}
Un altro fattore fondamentale è stata l'adozione di pratiche di DevOps e continuous delivery. Queste fanno fronte a necessità completamente nuove portate dalla filosofia cloud. Il loro scopo è fare in modo che si automatizzino tutti i processi come l'adozione del prodotto da parte di un nuovo cliente.

\paragraph{Per orientarsi nella consultazione di questa tesi}Per orientarsi nella consultazione di questa tesi andiamo a spiegare brevemente la sua struttura e le principali tematiche di trattazione. Nel capitolo 2 verrà descritto i contesto in cui si è svolta questa tesi in azienda. Verrà presentato il prodotto di IBM che si è portato su SaaS e verrà fatta una sulle metodologie di lavoro.
\paragraph{}
Nei capitoli 3 e 4 ci concentreremo sulla tecnologia del cloud computing e sulle sue diverse sfaccettature in cui si trova implementato. Un cambiamento importante come quello cloud comporta l'impiego di tecnologie apposite che verranno presentate nel capitolo 5.
\paragraph{}
Giungiamo così finalmente al capitolo 6 devo si espone in maniera molto approfondita il percorso progettuale che ha portato alla definizione dell'architettura del SaaS di BigFix. Quì si vedrà come la pratica del Design Thinking e del framework SCRUM ha portato a definire l'Architettura del prodotto che vede la sua implementazione descritta passo dopo passo nel capitolo 7.