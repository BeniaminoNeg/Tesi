\chapter{IBM BigFix}

\section{BigFix}
I prodotti della suite IBM BigFix consentono di monitorare e gestire in tempo reale un elevato numero di dispositivi fisici e virtuali connessi (fino a 300.000). Questi possono essere sia fisici che virtuali, come ad esempio server, desktop, notebook, dispositivi mobili, tablet, POS, ATM, chioschi self-service. Gli utenti principali di questi prodotti sono gli amministratori di sistema. Tramite le applicazioni BigFix possono avere il pieno controllo sugli endpoint, come distribuire software, applicare delle patch, effettuare il deploy di sistemi operativi, proteggere da attacchi di rete e molto altro.
\subsection{Architettura di BigFix}
L'architettura di bigFix si suddivide in due grandi macro-componenti, la platform e le applications. La prima svolge la funzione di layer sulla quale vengono sviluppate tutte le funzionalità dello strato di applications. Questa suddivisione consente una chiara suddivisione delle competenze da parte di progettisti, sviluppatori, tester e assistenti dei clienti. Il team della platform si concentra quindi nel fornire una solida infrastruttura al team delle applications, il quale svilupperà i singoli strumenti al servizio dell'utente.