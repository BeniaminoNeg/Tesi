\chapter{Conclusioni}

Questo lavoro di tesi ha visto la realizzazione, nel corso dei mesi, di un Software as a Service per l'azienda IBM. Una suite di servizi su cloud che permettesse di rivoluzionare i tool di IBM facenti parte della famiglia BigFix, una serie di prodotti per la security aziendale. Questi strumenti erano accessibili solamente nella tradizionale versione on premise. Il cliente doveva avere a disposizione una infrastruttura hardware notevole per poter permettere a BigFix di funzionare. Questo relegava il prodotto solo a clienti di grandi dimensioni che potevano permetterselo. 
\paragraph{}
Con la versione cloud di IBM BigFix, invece, l'unico componente che deve risiedere fisicamente presso il cliente è un agent installato su ogni endpoint fisico da controllare. Tutta la logica di controllo e le infrastrutture necessarie sono spostate presso il provider del servizio cloud, ovvero IBM stessa.

\paragraph{}
Per adottare questa nuova filosofia p stato necessario riprogettare completamente il prodotto. Il lavoro ha visto una corposa fase di design, coinvolgendo molti stackeholders come potenziali clienti, ma anche personale IBM come sviluppatori e futuri addetti all'assistenza. Adottando framework di progettazione e sviluppo come il "Design Thinking" e lo SCRUM con un processo agile iterativo e incrementale, si è definita un architettura dall'alto livello qualitativo e si è realizzato un prototipo di prodotto pronto per essere provato e messo sul mercato.
\paragraph{}
Si è scomposto il prodotto in un'architettura a microservizi e si sono utilizzate le più affermate tecnologie cloud come Docker e Kubernetes. E' stato fondamentale in tal senso adottare l'utilizzo di software container per rendere indipendenti i singoli servizi. In una seconda fase poi è stato necessario orchestrare tutte le componenti seguendo pratiche di DevOps e continuous integration

\section{Sviluppi futuri}
Il prodotto realizzato in azienda è, come detto prima, un prototipo. Questo verrà attentamente analizzato, presentato ai potenziali clienti e valutato. Se l'azienda lo riterrà opportuno, verrà finalmente inserito nel catologo e commercializzato dopo opportuni adattamenti. Occorrerà certamente raffinare l'architettura a microservizi tenendo ben presente la divisione delle resposabilità per ogniuno di essi. A regime si potranno aggiungere liberamente nuovi servizi a seconda delle necessità.
\paragraph{}
Un'altro importante aspetto è perseverare sulla via della continuous deliverry e della continuous integration. Queste pratiche faranno apprezzare maggiormente il loro contributo quando il prodotto verrà messo sul mercato. A quel punto i sempre più continui feedback degli utenti porteranno sempre a un progressivo miglioramento del servizio stesso.
\paragraph{}
Dal punto di vista delle tecnologie utilizzate, un importante aspetto sul quale si sta riflettendo è quello della possibilità di adottare a un database nativo cloud. Questa scelta marcherebbe una forte differenza con le tradizionali architetture dei database e consentirebbe di inegrare anche la persistenza in uno scenario di architettura cloud a microservizi.
\paragraph{}
Un prodotto aziendale di questo tipo è ovviamente aperto a tante possibili future evoluzioni. I feedabck e i futuri utilizzi del sistema porteranno sempre maggior conoscenza e consentiranno al prodotto di essere sempre più in grado di soddisfare le esigenze degli utenti.

\section{Considerazioni}

il priogetto èstato molto sfidante, impegnativo, ma....
