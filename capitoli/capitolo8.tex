\chapter{Conclusioni}

Questo lavoro di tesi ha visto la realizzazione, nel corso dei mesi, di un Software as a Service per l'azienda IBM. Una suite di servizi su cloud che permettesse di rivoluzionare i tool di IBM facenti parte della famiglia BigFix, una serie di prodotti per la security aziendale. Questi strumenti erano accessibili solamente nella tradizionale versione on premise. Il cliente doveva avere cioè a disposizione una infrastruttura notevole per poter permettere a BigFix di svolgere le proprie mansioni. Questo relegava il prodotto solo a clienti di grandi dimensioni che potevano permettersi ciò. 
\paragraph{}
Con la versione cloud di IBM BigFix, invece, l'unico componente che deve risiedere fisicamente presso il cliente è un agent installato su ogni endpoint fisico da controllare. Tutta la logica di controllo e le infrastrutture necessarie sono trasferite presso il provider del servizio cloud, ovvero IBM stessa.

\paragraph{}
Per adottare questa nuova filosofia è stato necessario riprogettare completamente il prodotto. Il lavoro è stato caratterizzato da una corposa fase di design, coinvolgendo molti stackeholders e potenziali clienti, ma anche personale di IBM come sviluppatori e futuri addetti all'assistenza. Adottando framework di progettazione e sviluppo come il "Design Thinking" e lo SCRUM, con un processo agile iterativo e incrementale, si è definita un'architettura dall'alto livello qualitativo e si è realizzato un prototipo di prodotto pronto per essere sperimentato e messo poi sul mercato.
\paragraph{}
Si è scomposto il prodotto con un'architettura a microservizi e si sono utilizzate le più affermate tecnologie cloud come Docker e Kubernetes. E' stato fondamentale, in tal senso, adottare l'utilizzo di software container per rendere indipendenti i singoli servizi. In una seconda fase poi è stato necessario orchestrare tutte le componenti contestualizzandole in pratiche di DevOps e continuous integration.

\section{Sviluppi futuri}
Il prodotto realizzato in azienda è, come detto prima, un prototipo. Questo verrà attentamente analizzato, presentato a potenziali clienti e valutato. Se l'azienda lo riterrà opportuno, verrà finalmente, dopo opportuni adattamenti, inserito nel catalogo e commercializzato. Occorrerà certamente raffinarne ulteriormente l'architettura a microservizi tenendo ben presente la divisione delle responsabilità per ognuno di essi. Una volta giunti a regime di produzione, si potranno aggiungere liberamente nuovi servizi a seconda delle necessità.
\paragraph{}
Un altro importante aspetto è perseverare sulla via della continuous delivery e della continuous integration. Queste pratiche faranno apprezzare maggiormente il loro contributo quando il prodotto verrà messo sul mercato. A quel punto i sempre più continui feedback degli utenti porteranno a un progressivo e definitivo miglioramento del servizio stesso.
\paragraph{}
Dal punto di vista delle tecnologie utilizzate, un'altra opportunità sulla quale si sta riflettendo è quella di adottare un database cloud nativo fornito da IBM Cloud. Questa scelta marcherebbe una forte differenza con le tradizionali architetture database e consentirebbe di integrare anche la persistenza in uno scenario di architettura cloud a microservizi.
\paragraph{}
Un prodotto di questo tipo, con il suo particolare mercato, è ovviamente soggetto a tante possibili future evoluzioni. I feedback e i futuri utilizzi del sistema porteranno continuamente maggior conoscenza e consentiranno al prodotto di essere sempre più al passo con le esigenze degli utenti.

\section{Considerazioni}
La realizzazione di questo servizio è stato un progetto molto sfidante e impegnativo, ma al tempo stesso mi ha visto affiancato da un team molto esperto e competente. Per questo è doveroso ringraziare tutto il team IBM di BigFix e in particolar modo l'ing. Bernardo Pastorelli per la sua disponibilità unita alla grande competenza. 
\paragraph{}
Il bagaglio di conoscenze acquisito durante questi mesi di lavoro è notevole sia sotto l'aspetto tecnico che dal punto di vista di gestione del ciclo di vita dei progetti. Al termine di questa importante esperienza è stato molto gratificante vedere apprezzato il lavoro di team in ambito dirigenziale aziendale sia a livello nazionale che internazionale. Ma soprattutto è stato ancora più appagante vedere come l'esperienza personale di tirocinio in IBM sia stata apprezzata e giudicata positivamente nella sede in cui si è lavorato.

\paragraph{}
Da un punto di vista tecnico, il prototipo si presenta già in uno stato molto avanzato. E' necessaria, a questo punto, solamente un'attenta indagine comprendendo se sono necessarie ulteriori modifiche al prodotto per renderlo definitamente pronto al suo utilizzo a livello industriale.
\paragraph{}
La strada intrapresa è sicuramente quella giusta. Il cloud computing rappresenta uno dei pilastri del futuro del software e orientarsi verso queste tecnologie significa aprirsi a nuovi scenari di utilizzo e a sempre più nuove opportunità. Non è fantascientifico immaginare che in un futuro molto prossimo ci si concerti solamente verso un utilizzo del software come servizio. Potrebbe non avere più senso detenere software e risorse localmente, ma ciò che farà la differenza sarà la fruizione delle risorse e dei prodotti. Ad esempio ci si potrebbe non preoccupare più di acquistare un personal computer con elevate specifiche hardware; basterà un prodotto che possa garantire un adeguato interfacciamento con risorse distribuite allocate dinamicamente. Risorse che in questo modo possono essere molto più prestanti e tecnologicamente avanzate.
\paragraph{}
In generale, in un mondo sempre più consumistico, in cui la maggior parte delle ricchezze sono sprecate, il cambio di prospettiva verso un utilizzo sapiente delle stesse e verso una sorta di "multi-tenancy" delle risorse comuni, può rappresentare una svolta considerevole.