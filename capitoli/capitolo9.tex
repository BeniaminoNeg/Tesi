\chapter{Ringraziamenti}

«Come facciamo a scoprire qual è la nostra missione?
Scalate la montagna che avete di fronte, attraversate il fiume davanti a voi.
Non possiamo sapere quale montagna scaleremo in futuro fino a quando non avremo superato quella che abbiamo di fronte.
Anche se questa	scelta vi fa soffrire, anche se non ne avete voglia o avete timore di ciò che possono pensare gli altri, scalate la montagna davanti a voi senza fuggire;
allora la vostra visuale si amplierà sempre di più.
Riuscirete a vedere molto lontano e capirete chiaramente la strada giusta da percorrere.
La distanza tra zero e uno è molto più grande di quella tra uno e cento.
Anche un viaggio di mille miglia inizia sempre da un primo passo».

\paragraph{}
Qualche anno fa mi è capitato di leggere questo discorso quando era il momento di scegliere quali fossero i primi passi da compiere. Ora che è terminato solo il primo dei sentieri, ripenso alla strada percorsa e mi rendo conto che non sono mai stato solo.

\paragraph{}
Ringrazio i miei nonni che mi hanno accompagnato già dai primissimi metri, che hanno trasmesso le mie radici che mi accompagneranno sempre.

\paragraph{}
In ogni ostacolo che ho incontrato davanti a me ci sono stati sempre i miei genitori ad affiancarmi per superare ogni difficoltà, a incoraggiarmi e supportarmi facendo sacrifici per non farmi mancare mai nulla. Si è unita presto a noi una piccola peste, portando un'allegra confusione... grazie sora!

\paragraph{}
Un ringraziamento va anche ai miei amici, con i quali ne abbiamo passate tante, ma alla fine siamo sempre tornati a casa sani e salvi, come se non fosse successo niente.

\paragraph{}
La strada percorsa mi ha portato a condividere il sentiero con compagni di classe, maestre e professori. Con loro ho scoperto le mie passioni interessi e inclinazioni accompagnandomi dal primo giorno di scuola alla maturità.

\paragraph{}
La mia strada mi ha portato a proseguire il mio percorso in questa università. Indubbiamente sono stati cinque anni formanti ma soprattutto tra i più divertenti! Dalla MyShop Srl con la quale conquisteremo il mondo agli approcci aggggili dei tre profeti e bulli vari... non ho trovato dei semplici colleghi di corso, ma una classe affiatata, con la quale svolgere i progetti è un vero divertimento. Non posso inoltre non ringraziare amici e professori che mi hanno accolto a Dublino, in una esperienza breve ma ricca di avventure e soddisfazioni.

\paragraph{}
Un doveroso ringraziamento va anche ai professori che ci hanno formato in questi anni, instaurando un rapporto di empatia che raramente si incontra in ambito universitario. In particolar modo un ringraziamento particolare va al professor Cicerone per avermi accompagnato nel percorso universitario facendomi appassionare all'ingegneria del software e per avermi accolto come suo relatore.

\paragraph{}
Il lavoro svolto in questo progetto di tesi non sarebbe mai stato possibile senza l'opportunità che mi è stata data dal team di IBM BigFix di Roma. Ho avuto la possibilità di essere accolto in una grande famiglia e per questo non posso non ringraziare Dante, Max, Elia, Leonardo, Luisa e tutto il team per il loro aiuto; Marco per essere stato un attentissimo tutor e Bernardo per avermi seguito con la sua eccezionale esperienza durante tutto lo stage.

\paragraph{}