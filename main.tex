\documentclass[hidelinks,a4paper,12pt]{tesiinfo}

%%Pacchetti utili anche se non necessari
\usepackage{amsfonts}

\usepackage{amsmath}

\usepackage{latexsym}


\usepackage{tabularx}
%% Supporto alla lingua italiana, dipende dal vostro ambiente latex
\usepackage[italian]{babel}
%% Supporto per i caratteri accentati
\usepackage[utf8]{inputenc}
%% Supporto alle figure
\usepackage{graphicx}
%% Supporto alle figure wrappate nel testo
\usepackage{wrapfig}
%% Supporto alle didascalie
\usepackage{caption}
%% Supporto alle figure composte
\usepackage{subfigure}
%% Supporto alle didascalie per figure composte
%%\usepackage{subcaption}
%% Supporto ai link esterni (URL)
\usepackage[bookmarks=true]{hyperref}

%% Supporto alla creazione di colonne multiple nel documento
\usepackage{multicol}
%% Supporto al comando \cite per BibTex
\usepackage{cite}
%% Supporto per i listati di codice
\usepackage{listings}
%% Supporto allo stile per il codice PHP creato da Nicola Sacco e Daniele Di Pompeo
\input{progrLang.sty}
%% Supporto al glossario
\usepackage[style=altlist, toc=true]{glossary} % can be obtained from http://www.ctan.org/tex-archive/macros/latex/contrib/glossary/
%% Richiesta di creazione del glossario
\makeglossary
%% Inclusione del file con i termini del glossario
\input{glossario/termini.tex}


\titolo{Realizzazione di un prototipo della versione Cloud SaaS della suite IBM BigFix: Automazione del deployment e del testing}
\laureando{Beniamino Negrini}
\relatore{Serafino Cicerone}
\annoaccademico{2016/2017}

%%Usare i seguenti comandi se si ha un correlatore:
\setcorrelatoreuno
\correlatoreuno{Dott. Marco Secchi}

%%Usare i seguenti comandi se si hanno due correlatori (NB: questi comandi sono alternativi a quelli precendenti):
%\setcorrelatoredue
%\correlatoreuno{Nome e cognome del primo correlatore}
%\correlatoredue{Nome e cognome del secondo correlatore}

%%Usare i seguenti comandi se si ha un relatore esterno (NB: questi comandi possono essere utilizzati con quelli precedenti):
\setesterno
\relatoreesterno{Dott. Bernardo Pastorelli}

%%Usare i seguenti comandi se si sta scrivendo una tesi di laurea specialistica
\setspecialistica

\begin{document}


\maketitle
   
  \begin{dedication}
  \textit{Dedica a piè pagina}
  \end{dedication}



\contentspage
 
  \chapter{CapitoloEsempio}
Capitolo introduttivo di prova
\begin{itemize}
 \item item di prova 1
 \item item di prova 2
\end{itemize}

\paragraph{Paragrafo}
 Paragrafo di prova
 \begin{figure}[h!]
	    \centering
	    \includegraphics[width=\textwidth,keepaspectratio=true]{capitoli/imgs/SystemDomain.png}
	    % CoreTotal.jpg: 3036x676 pixel, 72dpi, 107.10x23.85 cm, bb=0 0 3036 676
	    \caption{Immagine di prova}
\end{figure}
  \chapter{Introduzione}

\paragraph{}
 E' sempre più evidente che il cloud computing è il futuro del software. La rivoluzione consta nella distribuzione dei servizi di calcolo e alla virtualizzazione di un'utilizzo centralizzato. Tutto ciò è reso possibile dal momento in cui l'accesso alla rete è divenuto possibile da sempre più dispositivi e con velocità di connesione sempre maggiore.
  \chapter{IBM BigFix}

\section{BigFix}
I prodotti della suite IBM BigFix consentono di monitorare e gestire in tempo reale un elevato numero di dispositivi fisici e virtuali connessi (fino a 300.000). Questi possono essere sia fisici che virtuali, come ad esempio server, desktop, notebook, dispositivi mobili, tablet, POS, ATM, chioschi self-service. Gli utenti principali di questi prodotti sono gli amministratori di sistema. Tramite le applicazioni BigFix possono avere il pieno controllo sugli endpoint, come distribuire software, applicare delle patch, effettuare il deploy di sistemi operativi, proteggere da attacchi di rete e molto altro.
\subsection{Architettura di BigFix}
L'architettura di bigFix si suddivide in due grandi macro-componenti, la platform e le applications. La prima svolge la funzione di layer sulla quale vengono sviluppate tutte le funzionalità dello strato di applications. Questa suddivisione consente una chiara suddivisione delle competenze da parte di progettisti, sviluppatori, tester e assistenti dei clienti. Il team della platform si concentra quindi nel fornire una solida infrastruttura al team delle applications, il quale svilupperà i singoli strumenti al servizio dell'utente.
  \chapter{SaaS, le tecnologie che ne consentono la realizzazione}

\section{SaaS e i suoi requisiti}

\subsection{Availability e Reliability}

\subsubsection{Availability}
Il concetto di Availability, nel senso generale del termine, è ben definito dallo standard ITU-T E.800: "L'abilità di un sistema di essere in uno stato che soddisfa un determinato requisito, in determinati istanti di tempo, assumendo che le risorse a lui necessarie siano disponibili." Come possiamo vedere, è un concetto ben definito, ad ha quindi le sue metriche ben definite che quantificano l'Availability. 

\paragraph{MTTF, Mean Time TO Failure}
Misura l'intervallo di tempo tra due eventi di "faiulure", in cui il sistema non è riuscito a portare a termine il proprio compito.

\paragraph{}
Possiamo dire quindi che l'Availability rappresenta la porzione di tempo in cui il sistema si comporta secondo le proprie specifiche. Va tenuto in considerazione anche che, al verificarsi di un fallimento, al tempo di non-Availability si aggiunge il tempo per porre rimedio al fallimento e far ripartire il sistema. 

\subsubsection{Reliability}
La Reliability è definita anch'essa dalla International Telecommunications Union (ITU-T) recommendations E.800, come segue: "L'abilità di un sistema di soddisfare una funzione richiesta, sotto determinate condizioni e per un certo intervallo di tempo

\paragraph{}
Possiamo immaginare, a questo punto, quanto sia fondamentale un'altissima availability per i servizi Cloud. In caso di failure, infatti, possono potenzialmente essere tutti gli utenti serviti dal provider che ha subuto il guasto. I servizi erogati via Cloud dovrebbero essere disponibili da chiunque li richieda e da qualunque parte del mondo ventiquattro ore su ventiquattro. Ovviamente affidabilità massima non è verosimile, ma ci si aspetta una Reliability di molto vicina al 100. Ad esempio, BlueMix dichiara una Reliability del 99,95 percento. Per rendere l'idea, una Reliability del 99,95 percento sta a significare che, sulla base annuale, il servizio non è disponibile per circa 4 ore. La Reliabiity è un concetto affine all'Availability, con la differenza che la Reliability si riferisce all'abilità del sistema di compiere i suoi scopi durante un'intervallo di tempo. Essa infatti si quantifica con una probabilità.

\paragraph{Upgrade}
Sono gli scenari di Upgrade un'aspetto critico. Come si può immaginare, molti servizi SaaS hanno bisogno di essere continuamente aggiorati e modernizzati. All'uscita di una nuova versione del software occorre che questa venga distribuita a tutti gli utenti del servizio. Distribuire un software su scala Cloud non è semplice come si possa pensare. Non si può infatti interrompere l'erogazione del servizio per far partire il processo di agiornamento del software, che può essere più o meno lungo. Occorrerà quindi adottare delle tecniche che diano l'impressione all'utente di una ontinuità del servizio. Vedremo nei prossimi capitoli quale strategia abbiamo adottato con BigFix SaaS. 

\subsubsection{Dependability}


  \chapter{SaaS, le tecnologie che ne consentono la realizzazione}
Nel corso degli ultimi anni, con il proliferare delle piattaforme e dei servizi di cloud computing, sono nate e si sono sviluppate molte tecnologie per soddisfare le nuove esigenze e i nuovi requisiti appena visti che questa rivoluzione della fruizione del software ha comportato.

\section{Microservizi}
Un concetto fondamentale, di cui il Cloud Computing fa largamente uso, sono i mircoservizi. Cominciamo col darne una definizione abbastanza formale: "Lo stile architetturale a microservizi è un approccio allo sviluppo di una singola applicazione come insieme di piccoli servizi, ciascuno dei quali viene eseguito da un proprio processo e comunica con un meccanismo snello, spesso una HTTP API.(Martin Fowler)".

\paragraph{}
L'approccio è quello di dividere le funzionalità del sistema in più microservizi. Ad ogni microservizio corrisponde una necessità dell'utente. La filosofia di dividere il software in base alle responsabilità è già presente da tempo nell'ingegneria del software. Una suddivisione modulare del sistema in base ai casi d'uso dell'utente è un dogma dell'Object Oriented Design, ma la novità apportata dai microservizi è che il sistema risulta scomposto in piccoli servizi che sono completamente indipendenti tra loro. Ogni microservizio si preoccupa infatti di risolvere un particolare problema del cliente, un unico scenario. La comunicazione tra i servizi avviene attraverso la rete al fine di garantire l'indipendenza tra i servizi ed evitare ogni forma di accoppiamento. Ogni microservizio, infatti, rappresenta un'entità separata che generalmente può essere pubblicata come un modulo di una Platform as a Service.

\subsection{Il modello monolitico a layer }
Secondo il classico modello a layer le funzionalità vengono suddivise tra i diversi livelli in base al grado di astrazione, usando delle tecnologie proprie di ogni livello. Con questa architettura però, nonostante ci sia una suddivisione a strati, il software risulta essere un unico sistema monolitico, sebbene molti componenti possano essere comunque riusabili.

\begin{figure}[h!]
	\centering
	\includegraphics[width=\textwidth,keepaspectratio=true]{capitoli/imgs/disegnoMicrosMonol.png}
	\caption{Il modello monolitico e i microservizi}
\end{figure}

\subsection{Confronto tra l'architettura a microservizi e il modello monolitico}
La scelta di adottare uno o l'altro approccio viene dopo un'attenta analisi dei requisiti che il sistema deve soddisfare. In questo studio però va anche tenuto conto quanto le esigenze possano cambiare nei futuri utilizzi del software.
\begin{itemize}
	\item La struttura interna di tutto un sistema monolitico è composta dai layer di interfacciamento con l'utente, logica di business e persistenza dei dati. In un'architettura a microservizi non troviamo questa divisione a livello di sistema, ma la ritroviamo semmai all'interno di ogni singolo microservizio atomico. Il microservizio si occuperà, ad esempio, di preservare il suo stato tramite l'utilizzo di un proprio database non condiviso con gli altri microservizi. 
	
	\item Uno dei fattori chiave da considerare è la scalabilità. Per scalare un'applicazione monolitica occorre necessariamente clonarla in più server, macchine virtuali o contenitori.
	
	\item Quando occorre scalare orizzontalmente un'architettura a microservizi, si creano e si distribuiscono indipendentemente tra loro repliche dei microservizi in più server o container. 
\end{itemize}


\begin{figure}[h!]
	\centering
	\includegraphics[width=\textwidth,keepaspectratio=true]{capitoli/imgs/monosvmicro.png}
	\caption{Il modello monolitico e i microservizi}
\end{figure}

\subsection{Vantaggi e svantaggi di un'architettura a microservizi}
Un'architettura di questo tipo porta con se ovviamente anche i suoi svantaggi e i suoi vantaggi. Andiamo a vedere come questi ultimi corrispondano proprio alle esigenze del cloud computing.

\subsubsection{Vantaggi}
\begin{itemize}
	\item  Velocità \\
	L'architettura a microservizi è quella che si sposa meglio con la metodologia agile. un microservizio deve avere sempre delle dimensioni ridotte e il suo sviluppo ha una durata di circa due settimane. Ciò porta ad avere fin da subito piccole porzioni del sistema (i servizi appunto), pronte, testabili ed utilizzabili. Ogni microservizio inoltre è autonomo e può quindi giungere in ambiente di produzione indipendentemente dagli altri. In questo modo si riesce a reagire molto velocemente alle esigenze di mercato.
	
	\item Sperimentazione \\
	Con i microservizi la modularità del sistema è un notevole punto di forza. Sperimentare nuove tecnologie all'interno di un singolo microservizio ha un impatto nullo su tutti gli altri. Si è in questo modo molto più invogliati a ricercare sempre più nuove tecnologie da inserire nel proprio prodotto. Il rischio è minimo in quanto, anche nel caso l'esperienza risulti fallimentare, la mole di lavoro che comporta la modifica di un microservizio è veramente molto ridotta.
	
	\item Tecnologie ad hoc \\
	Altro fattore da considerare è la possibilità di differenziare le tecnologie a seconda del microservizio. Si prenda come esempio la vastità di tipologie di database che sono disponibili all'uso. Un database che è appropriato per un microservizio potrebbe non essere la scelta migliore per un altro. 
	
	\item Scalabilità \\
	I software con un'architettura a microservizi sono pensati per scalare orizzontalmente in maniera estremamente agevole. Si possono replicare a piacimento tramite l'utilizzo di containers e hanno un comportamento distribuito anche nel caso si trovino sulla stessa macchina, in quanto i container li isolano da tutti gli altri.
	
	\item Facilità di Deployment \\
	Le modifiche hanno un impatto molto ridotto nell'intero sistema. Grazie a ciò è possibile rilasciare sul mercato il software aggiornato con frequenze molto maggiori. Potenzialmente ogni modifica può subito essere pubblicata e non occorre attenersi a lunghi processi di release in cui si cerca il più possibile di accumulare modifiche da effettuare per poi applicarle tutte insieme.
	
	\item Portabilità \\
	Il software a microservizi è facilmente componibile e portabile su più contesti e dispositivi, come quello web, mobile ma anche sistemi embedded, dispositivi indossabili e molto altro.	
\end{itemize}

\subsubsection{Svantaggi}
\begin{itemize}
	\item Dipendenza dalla rete \\
	E' questo il principale punto di critico di un'architettura a microservices. Abbiamo visto quanto l'interazione tra i singoli microservizi faccia affidamento su una comunicazione attraverso la rete. Ovviamente questo deve essere un requisito fondamentale. In mancanza di una connessione adeguata tutto il sistema smette di funzionare correttamente.
	
	\item Identità e autenticazione \\
	Una volta che un utente effettua il login al sistema occorre garantire che la sua autenticazione, e soprattutto la sua identità, venga mantenuta in tutti i microservizi che andranno a comporre la sua esperienza utente.
\end{itemize}

\section{Containers}
I container sono l'habitat naturale dei microservizi. Essi forniscono al software tutto ciò che gli è necessario, garantendogli un ambiente estremamente leggero e flessibile.

\begin{figure}[h!]
	\centering
	\includegraphics[width=\textwidth,keepaspectratio=true]{capitoli/imgs/spostamentocontainer.png}
	\caption{Containers nell'accezione dell'industria dei trasporti}
\end{figure}

\paragraph{}
La parola container vuole alludere ad una analogia con l'attrezzatura specifica dei trasporti. Da dove è nata la necessità dell'utilizzo dei container? Immaginiamo lo scenario in cui alcune merci andassero trasportate prima via terra, magari con un camion, e poi via mare per raggiungere un altro stato. Prima dell'avvento dei container il camion arrivava al porto, andava aperto e il contenuto passato in un mercantile pronto sul molo. L'operazione era effettuata manualmente e i singoli imballaggi erano trasbordati uno alla volta con grande dispendio di tempo e mezzi. Si è iniziato allora a considerare più pratica l'idea di trasbordare sulla nave l'intero corpo del camion. Nacquero così i container: contenitori multiuso, realizzati in formati standard che possono essere passati con facilità da un camion a una nave e poi magari su un treno merci e così via.

\paragraph{}
E così anche nel nostro caso abbiamo bisogno di uno strumento che possa contenere i microservizi e le applicazioni, che ci consenta di manovrarle con facilità senza sapere cosa facciano esattamente.
I containers sono infatti un metodo di virtualizzazione del sistema operativo che ha come obiettivo quello di isolare il software che ospita, permettendo di eseguire le applicazioni e le loro dipendenze in processi completamente isolati. L'infrastruttura del container interagisce direttamente con il kernel della macchina che lo ospita, scavalcando gli altri layer. In qualsiasi sistema operativo collochiamo il container, le sue configurazioni interne rimarranno sempre separate, e l'applicazione contenuta si troverà garantito il suo ecosistema necessario all'esecuzione. Non ci si deve preoccupare, ad esempio, di produrre una versione software per Windows e una per sistemi UNIX, ma la soluzione a container è unica e portabile. 

\begin{figure}[h!]
	\centering
	\includegraphics[width=\textwidth,keepaspectratio=true]{capitoli/imgs/container.PNG}
	\caption{Schematizzazione di tre container sulla stessa macchina ospitante}
\end{figure}

\subsection{I containers e le Macchine Virtuali}
Stando alla descrizione dei container che abbiamo dato fino a questo punto potrebbe sorgere una domanda: perché utilizzare i container se potrebbero essere comodamente usate delle macchine virtuali? Una Virtual Machine è per sua natura già isolata dalla macchina che la ospita. La risposta sta nella leggerezza d'uso dei container. I container infatti non virtualizzano l'hardware della macchina, ma solamente il layer applicativo, rendendoli più portabili ed efficienti.

\paragraph{}
Non avendo un proprio sistema operativo, il peso di un container e dell'ordine di qualche Megabyte, contro i Gigabyte di una macchina virtuale che sovrappone il proprio sistema operativo a quello della macchina ospitante. A differenza delle macchine virtuali inoltre, i container non necessitano dell'Hypervisor, un componente che svolge delle attività di controllo e coordinamento sulle macchine virtuali. 

\begin{figure}[h!]
	\centering
	\includegraphics[width=\textwidth,keepaspectratio=true]{capitoli/imgs/ContainersvsVms.PNG}
	\caption{Confronto tra macchine virtuali e container, nell'esempio con l'utilizzo di un Docker Engine}
\end{figure}

\subsection{Vantaggi dei container}
Andiamo a ricapitolare quali sono le principali motivazioni che possono spingerci verso l'adozione dei container.
\begin{itemize}
	\item Coesione dell'ambiente \\
	L'ambiente di un container è fortemente disaccoppiato dalla macchina in cui si trova. Questo fa si che il proprio contenuto sia facilmente replicabile e portabile ovunque.
	
	\item Gestione delle risorse \\
	Con i container si ha una gestione delle risorse di calcolo molto più efficiente. Richiedendo poche risorse alla macchine ospitante, si ha la possibilità di eseguire molti più container contemporaneamente. 
	
	\item Produttività \\
	Diminuendo le dipendenze, ci si alleggerisce di tutta la mole di lavoro necessaria per configurare correttamente un prodotto. 
	
	\item Gestione degli aggiornamenti. \\
	Molto spesso la gestione degli aggiornamenti è un meccanismo già compreso nei container engine, come Docker.
\end{itemize}

\subsection{I container e i microservizi}
Possiamo quindi comprendere come i container siano estremamente appropriati ad ospitare dei microservizi. Ponendo un microservizio dentro un container, abbiamo la garanzia che ogni microservizio operi come un sistema separato, anche nell'eventualità che due servizi risiedano nella stessa macchina fisica. Ogni microservizio inserito in un container gode automaticamente di tutti i vantaggi di portabilità e flessibilità dei container che sono tra i requisiti fondamentali che ci spingono verso l'architettura a microservizi.

\subsection{Docker}

\begin{figure}[h!]
	\centering
	\includegraphics[width=\textwidth,keepaspectratio=true]{capitoli/imgs/docker.png}
	\caption{Logo di Docker}
\end{figure}

\paragraph{}
Docker è una piattaforma open source nata nel 2013 e scritta in linguaggio Go. Il suo scopo è quello di facilitare ed automatizzare il deployment delle applicazioni all'interno dei container. Il tool offre allo sviluppatore delle comode API di gestione che consentono agli sviluppatori di testare le applicazioni, effettuare build e distribuire il proprio prodotto. Docker fa utilizzo di numerose librerie come, ad esempio, libcontainer che ha lo scopo di interagire con il kernel di Linux.  In questo modo, isolando risorse, servizi e processi, dalla prospettiva dell'utilizzatore del container si ha l'impressione di un utilizzo esclusivo del sistema operativo.

\paragraph{}
Il container diviene con Docker l'unità di distribuzione del prodotto. Quando il servizio è stato correttamente implementato, è pronto per essere distribuito e, attraverso il container, può essere inserito in un orchestratore per garantirne il ciclo di vita.

\paragraph{Docker Engine}
L'Engine di Docker è una struttura a strati, troviamo:
\begin{itemize}
	\item Il server, un demone chiamato dockerd che è sempre in running. E' il componente che crea e gestisce gli oggetti Docker.
	\item Le REST API, che rappresentano l'interfaccia per interrogare il server.
	\item La Command Line Interface, che è quella con la quale si interfaccia l'utente e si aggancia alla REST API.
\end{itemize}

\begin{figure}[h!]
	\centering
	\includegraphics[width=\textwidth,keepaspectratio=true]{capitoli/imgs/dockerThinking.png}
	\caption{Docker engine}
\end{figure}

\paragraph{Architettura di Docker}
L'architettura di Docker è una classica architettura client-server. Il client parla direttamente con il Docker daemon che si trova sul server. E' proprio il dockerd che si prende l'onere di fare gran parte del lavoro, quello di far partire e distribuire i container.
\begin{figure}[h!]
	\centering
	\includegraphics[width=\textwidth,keepaspectratio=true]{capitoli/imgs/architecturedocker.png}
	\caption{Architettura di Docker}
\end{figure}
\paragraph{}
I componenti che vanno a costituire l'architettura di Docker sono i seguenti:
\begin{itemize}
	\item Docker deamon \\
	Abbiamo già parlato di lui. Dockerd è in ascolto di chiamate da parte della REST API. E' sua responsabilità far funzionare tutto l'engine della gestione dei container.
	\item Docker client \\
	Lo scopo principale del client è interagire con chi sviluppa il contenuto dei container. Questo componente recepisce i comandi da parte dello sviluppatore e li traduce in chiamate per il deamon.
	\item Docker registries \\
	Sono i registri che consentono di conservare le immagini di Docker. Immaginiamo che sia interesse di chi li sviluppa poter conservare e mettere in vendita servizi implementati all'interno dei container. Vengono offerte anche delle comode API e uno store nel quale mettere a disposizione le proprie immagini.
	\item  Docker objects \\
	In questa categoria rientrano più tipologie di oggetti Docker, come quelli che andiamo ad analizzare qui di seguito. \begin{itemize}
		\item  Images \\
		Un'immagine è un template standard che si utilizza per parametrizzare un nuovo container che si sta per mettere in piedi. Per creare la propria immagine, con i parametri personalizzati, si dichiarano nel Dockerfile gli step necessari. Successivamente ad ogni step corrisponderà a un layer del container.
		\item  Container \\
		Un container è l'istanziazione di una immagine, esso infatti è definito univocamente dalla propria immagine. E' un elemento che può essere gestito a piacimento e può essere collegato ad una o più reti, fattore importantissimo per far interagire il proprio contenuto con quello di altri container.
		\item Services \\
		I servizi permettono di scalare i container attraverso più Docker deamon. In questo modo si riesce a dare ugualmente l'impressione all'utente di utilizzare un servizio esclusivo e centralizzato.
	\end{itemize}
\end{itemize}

\subsection{Kubernetes}
\begin{figure}[h!]
	\centering
	\includegraphics[width=\textwidth,keepaspectratio=true]{capitoli/imgs/kubernetes_full.png}
	\caption{Logo di Kubernetes}
\end{figure}

\paragraph{}
Abbiamo appena tessuto le lodi di Docker. Ora però poniamoci alcune domande: come possono essere coordinati, distribuiti e gestiti i container e il loro workload, man a mano che vengono consumate le risorse disponibili nell'infrastruttura sottostante? Come operano i container in un ambiente network multi-tenant? Quale livello di sicurezza propone Docker? E chi decide quale sia il giusto livello di astrazione?

\paragraph{}
Per far fronte a queste necessità è nato da Google nel 2014 Kubernetes. Esso offre un layer di astrazione per migliorare le performance dei container stessi eliminando molti dei processi manuali coinvolti nel deployment e nella scalabilità di applicazioni containerizzate. Consente di far cooperare opportunamente insiemi di container componendo delle unità logiche utili nelle nostre applicazioni.

\paragraph{Perché utilizzare Kubernetes?}
Le applicazioni di produzione si espandono spesso su più container, questi devono essere distribuiti a loro volta su diversi server host. Kubernetes offre le capacità di orchestrazione e gestione necessarie per distribuire i container, in modo scalabile, al fine di gestire i carichi di lavoro. L'orchestrazione di Kubernetes consente di creare servizi che si estendono su più container, gestirne la scalabilità e l'integrità nel tempo.

\begin{figure}[h!]
	\centering
	\includegraphics[width=\textwidth,keepaspectratio=true]{capitoli/imgs/kubernetes-diagram.png}
	\caption{Ruolo di orchestratore di Kubernetes}
\end{figure}

\paragraph{I Pod di Kubernetes, cosa sono?}
Kubernetes risolve molti dei noti problemi relativi alla proliferazione dei container, raggruppandoli in un pod. I pod aggiungono quindi un livello di astrazione ai cluster di container. La loro funzione è quella di alleggerire i carichi di lavoro e fornire i servizi necessari, tra cui rete e storage, ai container stessi. Kubernetes agevola, inoltre, il bilanciamento del carico all'interno dei pod e garantisce l'utilizzo di un numero di container adeguato per supportare i carichi di lavoro.

\paragraph{Architettura di Kubernetes}
\begin{figure}[h!]
	\centering
	\includegraphics[width=\textwidth,keepaspectratio=true]{capitoli/imgs/kubernetesarchitetcture.png}
	\caption{Architettura di Kubernetes}
\end{figure}

Andiamo a definire quelli che sono i termini di riferimento in Kubernetes:
\begin{itemize}
	\item Master \\
	E' la macchina che controlla i nodi Kubernetes. È il punto di origine di tutte le attività assegnate.
	\item Nodi \\
	Queste macchine eseguono le attività assegnate richieste. Sono controllate dal nodo master di Kubernetes. Possono essere assimilate con gli host dei container.
	\item Pod \\
	Rappresenta un gruppo di uno o più container distribuiti su un singolo nodo. Tutti i container presenti in un pod condividono indirizzo IP, IPC, nome host ed altre risorse. I pod astraggono la rete e lo storage dal container sottostante, consentendo di spostare i container nei cluster con maggiore facilità.
	\item Service \\
	Questa componente ha la funzionalità di disaccoppiare le definizioni del lavoro dai pod.
	\item Kubelet \\
	Questo servizio viene eseguito sui nodi, legge i manifest del container e garantisce che i container definiti vengano avviati ed eseguiti.
	\item Kubctl \\
	Questo componente si interfaccia direttamente con il programmatore. Presenta una riga di comando per la creazione e gestione dei pod.
\end{itemize}

\paragraph{E Docker?}
La piattaforma docker mantiene le proprie funzioni. Quando Kubernetes assegna un pod ad un nodo, il kubelet su quel nodo chiede a docker di lanciare i container specificati. Quindi, il kubelet legge continuamente lo stato di quei container da docker e aggrega le informazioni nel nodo master. Docker invia i container sul nodo e avvia e arresta opportunamente i container. La differenza con lo scenario senza Kubernetes è che un sistema automatizzato con Kubernetes chiede a docker di eseguire queste operazioni, anziché assegnarle ad un amministratore, il quale deve eseguirle manualmente su tutti i nodi, in tutti i container.

\section{Multitenancy}
La multitenancy è un paradigma che, andando a braccetto con il cloud computing, si sta diffondendo sempre di più nel panorama delle architetture software. Esso prevede che una singola istanza del software in questione, situato su un server, sia utilizzato da molti utenti, chiamati tenant. I tenant condividono l'accesso comune al servizio con privilegi differenziati sulle istanze software. L'obiettivo principale è quello di garantire scalabilità al servizio e dare al tenant la percezione di un possesso esclusivo del software.
\begin{figure}[h!]
	\centering
	\includegraphics[width=0.7\textwidth,keepaspectratio=true]{capitoli/imgs/multitenancy.png}
	\caption{Esempi di Multitenancy e Single-Tenancy per il contesto abitativo}
\end{figure}

\paragraph{Vantaggi della Multitenancy}
\begin{itemize}
	\item Costi \\
	Scalare un'architettura multitenant è più economico, in quanto si interviene solamente in un'unica macchina fisica e acquistare hardware più performante risulta meno dispendioso. Inoltre in questo modo si riduce anche il numero di licenze da acquistare, nel caso siano necessarie.
	\item Data Mining \\
	Nel caso sia necessario estrapolare dei dati da tutti i clienti, avere dei tenant tutti sulla stessa macchina rende l'operazione molto più immediata.
	\item  Release Management \\
	Al rilascio di una nuova versione il processo risulta essere molto semplificato. Ciò viene ovviamente dal fatto che dovendo aggiornare una sola istanza del software su una sola macchina fisica la mole di lavoro è sicuramente minore.
\end{itemize}

\paragraph{Requisiti}
Ovviamente un'architettura multitenant ha anche dei requisiti. Nonostante l'unicità del software, si deve garantire un alto grado di "personalizzazione" ai diversi tenant, proprio come se avessero un utilizzo esclusivo del servizio. Da un altro lato ci si aspettano elevati parametri di security, robustezza e prestazioni.

\section{DevOps, Continuous Delivery e Continuous Integration}

\section{Monitoring Tools}
La centralizzazione portata dal SaaS comporta anche un radicale cambiamento delle modalità di monitoring del sistema. Non è pensabile che una persona possa andare a controllare tutti i log dei servizi e misurare le prestazioni degli stessi manualmente. E' necessario l'utilizzo di appositi tool che hanno come finalità quella di raccogliere questa mole di informazioni, elaborarle e presentarle al personale addetto del provider nel modo più intellegibile possibile. In questa ottica andiamo a presentare due tool utilizzati nel progetto BigFix SaaS: Prometheus e Grafana.
\subsection{Prometheus}
\begin{figure}[h!]
	\centering
	\includegraphics[width=0.5\textwidth,keepaspectratio=true]{capitoli/imgs/prometheuslogo.png}
	\caption{Logo di Prometheus}
\end{figure}
\paragraph{}
Prometheus è un tool di monitoring open-source nato nel 2012 e scritto in Go. Si adatta sia alle architetture centralizzate che a quelle distribuite. La struttura del tool è pensata per garantire un validissimo supporto nel caso il servizio monitorato abbia dei malfunzionamenti. Vediamo un elenco delle caratteristiche principali.
\begin{itemize}
	\item Un data model multidimensionale che ha lo scopo di raccogliere tutti i dati.
	\item Un query language flessibile per far fronte a questi dati.
	\item Engine per la rielaborazione dei dati.
	\item Meccanismi sofisticati per collezionare serie storiche.
	\item Supporto per le dashboard.
\end{itemize}
\subsection{Grafana}
\begin{figure}[h!]
	\centering
	\includegraphics[width=0.3\textwidth,keepaspectratio=true]{capitoli/imgs/grafanalogo.png}
	\caption{Logo di Grafana}
\end{figure}
Grafana consente di avere una finestra da cui monitorare tutti i servizi in maniera centralizzata. Consente di visualizzare dati, creare degli alert e visualizzare tutte le informazioni in comode interfacce per gli utenti.
\begin{figure}[h!]
	\centering
	\includegraphics[width=0.7\textwidth,keepaspectratio=true]{capitoli/imgs/grafanainterface.PNG}
	\caption{Una delle interfacce di Grafana}
\end{figure}
\section{BlueMix Services}
\begin{figure}[h!]
	\centering
	\includegraphics[width=0.7\textwidth,keepaspectratio=true]{capitoli/imgs/bluemixlogo.png}
	\caption{Logo di IBM BlueMix}
\end{figure}
Come abbiamo accennato precedentemente, IBM BlueMix è una PaaS che offre molti microservizi utili per il cloud computing. Il vantaggio dei microservizi è che possono essere utilizzati a piacimento in qualsivoglia progetto, come nel nostro caso. Alcuni dei servizi BlueMix sono stati anche utilizzati nel nostro progetto e altri lo saranno in seguito, con degli sviluppi futuri.
\paragraph{}
Ne sono un esempio il database DB2 e Kubernetes, che nel nostro caso sono stati integrati utilizzando le loro versioni presenti su BlueMix. E' possibile inoltre che in futuro si adotti un'altro database IBM sempre presente su BlueMix, Cloudant, un database NoSQL nativo cloud.

  \chapter{IBM BigFix on SaaS, la progettazione}
Per la realizzazione di un prototipo di questo tipo, è necessaria un'attenta progettazione. Quella di cui andremo a parlare a breve è la fase dell'ideazione che va dall'identificazione dei requisiti funzionali, dei bisogni dell'utente, alla definizione dei parametri qualitativi che deve avere il prodotto finale, determinando scelte molto importanti dal punto di vista architetturale. Al termine di questa fase il team inizia gli sprint di development, raffinando sempre di più il modello del servizio che si vuole realizzare. Andiamo ora a vedere quali sono i primi passi che si sono mossi nella realizzazione del progetto.

\section{Interaction Design}
Non si può prescindere dal fatto ce l'interaction design sia la primissima fase da affrontare all'inizio del progetto. Se non si parte dai reali bisogni dell'utente finale, si rischia inevitabilmente di sbagliare strada e realizzare un prodotto che non avrà mai successo sul mercato. Nell'affrontare questo tipo di progettazione l'IBM ha adottato un framework sempre pù diffuso nel panorama dello studio dell'usabilità, il Design Thinking.

\subsection{Design Thinking}
Empatia. E' questa la parola chiave della filosofia del Design Thinking. E' un processo creativo che ha come scopo quello di mettere al centro del progetto le necessità dell'utente. Ma proprio per meglio comprendere queste necessità è indispensabile instaurare un rapporto di empatia con gli utenti stessi. Capire i loro reali bisogni, ma anche osservarli durante la loro vita quotidiana per comprendere quelle necessità che non vengono direttamente esternate. Al tempo stesso si vogliono massimizzare le occasioni di feedback cercando di produrre il prima possibile degli output che permettano di comparare più soluzioni alternative. Aumentando per quanto possibile gli input, si riducono le probabilità di fallimento. Ma analizziamo i diversi step che accompagnano un percorso di design thinking.
\begin{figure}[h!]
	\centering
	\includegraphics[width=\textwidth,keepaspectratio=true]{capitoli/imgs/Design-Thinking.png}
	\caption{Schematizzazione molto basilare del Design Thinking}
\end{figure}
\begin{itemize}
	\item Definire il problema \\
	In questa fase del lavoro è fondamentale osservare, capire le abitudini degli utenti ed immedesimarsi in loro.
	\item Divergere \\
	Quì entra in gioco la creatività, cercando di mettere sul piatto il maggior numero di soluzioni possibili, evitando però preconcetti sulle soluzioni e concentrandosi solamente sul problema.
	\item Testare \\
	E' necessario ora produrre dei prototipi testabili in modo che si possano comparare le diverse soluzioni e ricevere feedback dagli utenti.
	\item Convergere \\
	A questo punto è il tempo di dirigersi verso una soluzione, utilizzando anche la creatività per attuare dei compromessi tra quelle soluzioni parziali che intersecano al meglio i requisiti.
\end{itemize}


\subsection{BigFix SaaS Interaction Design}
Il nodo cruciale di questa fase del lavoro e stato per noi quello di capire realmente quale fosse il target del nuovo servizio SaaS e quali siano i reali bisogni che possano spingere i clienti ad adottare un prodotto SaaS, siano essi già degli utenti della versione on premise o no. Il problema principale prima dell'avvento del paradigma design thinking era che i requisiti funzionali dei prodotti che venivano realizzati per le aziende erano stabiliti tramite delle contrattazioni svolte tra i progettisti e gli addetti agli acquisti delle aziende clienti, spesso trascurando i reali beneficiari del prodotto, ossia i tecnici dell'azienda cliente. Questo spesso porta a realizzare dei prodotti che non fanno fronte ai reali bisogni dell'utente. 


\paragraph{Stakeholder Map}
L'obiettivo principale di questa fase del lavoro è stata per noi quella di allineare tutti gli interessati, sviluppatori, dirigenti e potenziali utenti, sugli obiettivi del progetto SaaS. Per fare ciò si è fatto uso della Stakeholder Map, un artefatto che raffigura per l'appunto tutti gli Stackeholder interessati alla realizzazione del progetto. 
\begin{figure}[h!]
	\centering
	\includegraphics[width=\textwidth,keepaspectratio=true]{capitoli/imgs/StakeholderMap.PNG}
	\caption{BigFix on SaaS Stakeholder Map}
\end{figure}
Possiamo notare come gli input vengano da figure dirigenziali, che dettano le direttive aziendali, e da clienti del panorama cloud. E'inoltre necessaria una stretta interazione tra il team di sviluppo e il Security Operation Team, ovvero il team che dovrà garantire la manutenzione del servizio SaaS, monitorando le prestazioni del servizio e intervenendo se necessario.

\paragraph{User Research}
Il coinvolgimento della figura dell'utente finale è avvenuto fin dalla progettazione. Questo è stato fatto tramite interviste strutturate, ma anche osservando gli utenti nella loro routine lavorative, cercando di captare necessità e frustrazioni che vogliono essere eliminate e annotarle. L'obiettivo di fondo di questa fase del lavoro era quello di instaurare un rapporto di empatia tra gli utenti e chi realizza il progetto.

\paragraph{Nascita delle personas}
A questa punto occorreva fare un'operazione di astrazione. Cercare di identificare degli elementi chiave dal lavoro precedente e impersonificare i bisogni e le caratteristiche scoperte in delle personas. Le personas sono personaggi fittizi che vengono creati per rappresentare i diversi tipi di utenti in base alle loro caratteristiche comportamentali. Vengono utilizzate per creare degli scenari e capire meglio il target del lavoro che si sta per compiere. Nel nostro caso sono state individuate le seguenti personas:
\begin{itemize}
	\item Rick - BigFix Operator \\
	Rick è l'amministratore che utilizza BigFix nella sua versione SaaS. Vuole poter festire gli endpoint della sua azienda come farebbe con la versione on premise, distribuire contenuti e forzare le policy aziendali. usa le applications di BigFix.
	\item James - Content Creator \\
	James crea i contenuti per BigFix, crea Fixlet e task ad hoc per la propria azienda e crea pacchetti da deployare su diversi endpoint aziendali.
	\item Scott - BigFix Architect \\
	Scott è l'architetto dell'azienda cliente. Si occupa di installare la suite BigFix, che nel caso della versione SaaS risulta essere molto più agevole. Deve stabilire quale sia l'architettura azienda aziendale, relay e la rete di agent.
	\item  Rafael - Security Analyst \\
	Rafael è la figura che si occupa di controllare che gli endpoint rispettino le policy aziendali e può mandare dei messaggi per sollecitare l'attuazione della compliace.
	\item Lucy - IT Manager \\
	Si occupa della gestione dell'infrasruttura IT dell'azienda. Ha bisogno di accedere a molti contenuti dell'operator e si interfaccia spesso con Rick.
	\item Hugo - OPS engineer \\
	Hugo è un dipendente IBM che si occupa della manutenzione del servizio SaaS.
\end{itemize}

\paragraph{Empathy Map}
A questo punto è stato necessario definire gli aspetti caratteriali dal punto di vista lavorativo delle personas appena individuate. Questo si formalizza attraverso una Empathy Map per ogni peronas che si vuole analizzare. In questo artefatto vengono poste al centro e ne vengono appuntate le peculiarità. Abbiamo scritto cosa fa durante la giornata e le sue necessità, ma anche dopo un'analisi quelli che possono essere i suoi sentimenti durante lo svolgimento dei task quotidiani. tra le caratteristiche che abbiamo cercato di delineare nei personaggi ci sono:
\begin{itemize}
	\item Profilo professionale
	\item Attività
	\item Attitudini
	\item Bisogni
	\item Obiettivi
	\item Cosa possiamo fare per aiutarlo
\end{itemize}
\begin{figure} [h!]
	\centering
	\includegraphics[width=0.7\linewidth]{capitoli/imgs/empatymap}
	\caption{Esempio di Empathy Map}
	\label{fig:empatymap}
\end{figure}
\begin{figure} [h!]
	\centering
	\includegraphics[width=0.7\linewidth]{capitoli/imgs/HugoEM.PNG}
	\caption{Empathy Map di una delle nostre personas}
	\label{fig:hugoem}
\end{figure}

\paragraph{Scenario Map}
In questo nuovo elaborato le personas con le quali siamo entrati in confidenza nella fase precedente iniziano ad essere inserite nel loro ciclo di vita quotidiano. Con lo scenario viene descritto un flusso di lavoro tipico del personaggio in questione. Nel succedersi degli stages, ovvero i passi in cui si divide lo scenario, si annotano quelli che possono essere i sentimenti dell'utente, cercando di individuare gli elementi di frustrazione e le opportunità per intervenire nella progettazione risolvendo i problemi dell'utente tipo. Vediamo qui di seguito una scenario map per Rafael.
\begin{figure} [h!]
	\centering
	\includegraphics[width=0.7\linewidth]{capitoli/imgs/scenarioRafael.PNG}
	\caption{Scenario che rappresenta il primo utilizzo del servizio da parte di Rafael, il Security Analist }
	\label{fig:scen1}
\end{figure}

\paragraph{}
Vediamo quì di seguito uno schizzo riassuntivo degli step del Design Thinking che abbiamo adottato:
\begin{figure} [h!]
	\centering
	\includegraphics[width=0.7\linewidth]{capitoli/imgs/schizzodesthink.PNG}
	\caption{Step del Design Thinking adottati nel nostro progetto}
	\label{fig:dt}
\end{figure}

\paragraph{}
Gli scenari in questo modo individuato vanno a rappresentare la base per la stesura delle Epiche. Queste, come abbiamo descritto nella sezione 2.3 verranno poi raffinate con la stesura delle Storie e infine divide nei Task implementabili dagli sviluppatori.
\begin{figure} [h!]
	\centering
	\includegraphics[width=0.7\linewidth]{capitoli/imgs/scenario.PNG}
	\caption{Relazione tra gli scenari individuati in questa fase e le Epiche del framework SCRUM}
	\label{fig:scentospic}
\end{figure}

\section{Requisiti Non Funzionali}
Abbiamo già parlato nel capitolo 4 di quali sono le nuove problematiche alle quali una SaaS application deve far fronte. Ovviamente nel mio lavoro di tesi questo aspetto è stato un argomento cruciale delle prime fasi del lavoro. Soddisfare questo tipo di requisiti comporta infatti fare scelte architetturali molto impattanti e in quanto tali occorre definirle prima possibile nel design di un sistema software. 

\subsection{Dependability}
Il servizio di BigFix SaaS è stato progettato per garantire, quando sarà in produzione, un'availability che si mantenga sempre su valori superiori al 99. Ovviamente si prevedono carichi di utilizzo che possono essere anche molto elevati. La suite di BigFix è utilizzata contemporaneamente da clienti di tutto il mondo, alcuni dei quali possiedono una rete di endpoint composta da un numero considerevole di nodi. Tutto ciò può portare a picchi di carico molto elevati nonostante i quali il servizio deve continuare a essere disponibile con prestazioni sopra delle soglie minime di accettabilità.

\paragraph{Microservizi e container}
Come abbiamo potuto osservare nei capitoli precedenti, l'adozione di microservizi e container è un must per i servizi cloud. Grazie a questa scelta possiamo garantire agli utenti di BigFix SaaS un'alta Dependability, fattore fondamentale nel contesto della security aziendale in cui si va a calare questa suite di prodotti. I microservizi di BigFix, infatti, verranno replicati tramite i container in datacenter IBM in tutto il mondo, ciò potrà garantire anche tolleranza ai guasti che possono presentarsi. Il grado di replicazione dei diversi microservizi sarà ovviamente proporzionale all'importanza del microservizio stesso. Ci saranno ovviamente dei microservizi con dei ruoli più centrali di altri.

\subsubsection{Rolling Update}
Un'altro aspetto critico nel garantire un'alta availability è quello dell'aggiornamento del servizio. Facendo un paragone con i servizi SaaS che utilizziamo quotidianamente per consultare la posta elettronica, notiamo che non assistiamo mai a fenomeni di mancanza del servizio quando il prodotto si aggiorna, ma, all'occorrenza, troviamo già il prodotto nella sua versione agiornata. Vogliamo che questo comportamento si verifichi anche con la suite SaaS di BigFix e per questo occorre attuare una politica di Rolling Update. Silentemente, vengono aggiornate a turno tutte le repliche dei microservizi interessanti dall'aggiornamento. Nel fare ciò però, l'esperienza utente non risente di peggioramenti, in quanto le repliche che rimangono in servizio garantiscono l'efficienza del servizio.

\subsubsection{Utilizzo di BD2}
Anche la persistenza dei dati può risultare essere un elemento critico per la dependability. Occorre uno strumento che garantisca l'integrità dei dati, la resistenza ai guasti con adeguate misure di ripristino e soprattutto la riservatezza dei dati che, in un contesto come la security aziendale, possono essere molto sensibili. Si è scelto di utilizzare come DBMS DB2, un database relazionale prodotto da IBM. Una peculiarità di questo prodotto è la HADR (High Availability and Disaster Recovery). Diamo un'occhiata all'architettura di DB2 per capire di cosa si tratta.

\begin{figure}[h!]
	\centering
	\includegraphics[width=\textwidth,keepaspectratio=true]{capitoli/imgs/db2.PNG}
	\caption{Architettura del DBMS IBM DB2}
\end{figure}

\paragraph{}
DB2 replica tutto il contenuto del suo database, chiamato Primary Database, in un secondo database detto Standby Database, il quale svolge anche il ruolo di backup. I dati di questi due sono consistenti e vengono sincronizzati costantemente. Qualunque malfunzionamento del database principale normalmente comporterebbe dei tempi di non availability più o meno lunghi. Con questa architettura HADR, invece, nel momento in cui il Primary Database presenta un guasto, lo Standby Database assume il suo ruolo (Failover) finchè il database primario non torna disponibile e, a qual punto, i due database tornano a svolgere il loro compito originario (Role switch). Una prerogativa importante però è che i due database risiedano in due data center distinti, o comunque provengano da due fonti di energia distinte nel caso si trovino nello stesso luogo geografico, per evitare che dei guasti possano colpirli entrambi. 

\subsection{Scalabilità}
Per quanto riguarda la Scalabilita ci prefiggiamo di garantire le stesse specifiche del prodotto in versione on premise, quindi di supportare fino a 250.000 endpoint per server. Il soddisfacimento di questa specifica, nel contesto SaaS, sposta l'attenzione ovviamente sul nuovo concetto di server, ossia una serie di microservizi distribuiti che svolgono le funzionalità che nella versione on premise era svolta dal server presso il client. Ancora una volta sta nella ridondanza dei microservizi la chiave per garantire la scalabilità prefissata.

\subsection{Monitoring}
Sotto l'aspetto del monitoring ci siamo dovuti scontrare con una nuova complessità nel saper monitorare un servizio così diffuso come quello di SaaS. La necessità è quella di sostituire l'intervento umano nella consultazione dei log di tutti i servizi. Il requisito che abbiamo è quello di analizzare i risultati, saper effettuare delle medie e calcolare dei picchi di parametri come il throughput o la latenza. Vogliamo infine che questa mole di dati fosse facilmente consultabile agli occhi di chi effettua la manutenzione del prodotto, magari sotto forma di grafici facilmente intellegibili. Per soddisfare queste necessità abbiamo individuato i tool Prometheus e Grafana che si sono rivelati molto utili nelle fasi successive al deployment, come vedremo in seguito.  


\section{Gap con il prodotto on premise}
La natura di un servizio SaaS porta con se alcune differenze strutturali importanti con il prodotto già esistente. La modalità di fruizione del prodotto è completamente diversa dal prodotto on premise infatti e gli accorgimenti sono da prendere subito in considerazione in quanto impattano pesantemente sulle scelte architetturali.
\paragraph{Introduzione della multitenancy}
Uno di questi è sicuramente la multitenancy. Nel modello SaaS può capitare che sulla stessa macchina fisica risiedano più server di clienti diversi. Dalla prospettiva utente però si deve dare l'impressione di un possesso esclusivo del server tramite strategie di multitenancy. E' di fondamentale importanza che un cliente non entri in contatto con dati afferenti al server di altre organizzazioni, anche se queste risiedono sullo stesso server fisico. Tra gli accorgimenti attuati c'è la modifica della modalità di archiviazione dei dati, permettendo di filtrare i dati appartenenti al tenant corretto e speciali privilegi di utilizzo dei servizi server. 
\paragraph{Introduzione dei microservizi}
L'introduzione dei microservizi è un elemento centrale della conversione a SaaS. Per attuarla è necessario un attento percorso di refactoring del codice del prodotto, suddividendolo in servizi coesi che possano rappresentare delle entità separate che cooperino tra loro.

\section{Definizione Architetturale}
La definizione di un'architettura di un prodotto così complesso è guidata da un'attenta analisi dei requisiti appena descritti. Occorre definire componenti ben definiti da implementare, altri da riutilizzare, alcuni da adattare e inoltre interfacciarsi con nuove tecnologie che devono essere opportunamente inserite nel contesto di applicazione.
\subsection{Scelta dei tool e dei servizi da utilizzare}
\subsection{Viste architetturali}

\section{Definizione dei processi di gestione}
\subsection{Novità rispetto al prodotto già esistente}
\subsection{Disaster Recovering}


\bibliografia{bibliografia/bibliografia}{}

\appendice
  \printglossary
  \chapter{Tecnologie Utilizzate}

\section{Linguaggi di programmazione}
\begin{itemize}
 \item Bash	\\
 \href{https://www.gnu.org/software/bash/}{https://www.gnu.org/software/bash/};
 \item Java 8\\
 \href{https://java.com/}{https://java.com/};
 \item C++\\
 \href{http://www.cplusplus.com/}{http://www.cplusplus.com/};
 \item Sed\\
 \href{http://www.grymoire.com/Unix/Sed.html}{http://www.grymoire.com/Unix/Sed.html};
\end{itemize}

\section{Linguaggi di Markup e di serializzazione}
\begin{itemize}
 \item YAML\\
 \href{http://yaml.org/}{http://yaml.org/};
 \item XML;
\end{itemize}

\section{Tool di sviluppo}
\begin{itemize}
 \item Docker \\
\href{https://www.docker.com/}{https://www.docker.com/};
 \item Kubernetes\\
\href{https://kubernetes.io/}{https://kubernetes.io/};
 \item Jenkins\\
\href{https://jenkins-ci.org/}{https://jenkins-ci.org/};
 \item Ansible\\
\href{https://www.ansible.com/}{https://www.ansible.com/};
\item IBM UrbanCode Deploy\\
\href{http://www-03.ibm.com/software/products/it/ucdep}{http://www-03.ibm.com/software/products/it/ucdep};
\item Prometheus\\
\href{https://prometheus.io/}{https://prometheus.io/};
\item Grafana\\
\href{https://grafana.com/}{https://grafana.com/};
\item IBM BlueMix (IBM Cloud)\\
\href{https://www.ibm.com/cloud/}{https://www.ibm.com/cloud/};
\item JUnit\\
\href{http://junit.org/junit5/}{http://junit.org/junit5/};
\item JUTAA;\\
\item Clair\\
\href{https://github.com/coreos/clair}{https://github.com/coreos/clair};
\item IBM AppScan\\
\href{https://www.ibm.com/security/application-security/appscan}{https://www.ibm.com/security/application-security/appscan};
\item VMware\\
\href{https://www.vmware.com/it.html}{https://www.vmware.com/it.html};
\end{itemize}

\section{Framework agili}
\begin{itemize}
\item SCRUM\\
\href{https://www.scrumalliance.org/}{https://www.scrumalliance.org/};
\item DevOps;\\
\item Design Thinking\\
\href{https://www.ibm.com/design/thinking/}{https://www.ibm.com/design/thinking/};
\end{itemize}

\section{Code Editors}
\begin{itemize}
	\item Vim\\
	\href{https://github.com/vim/vim}{https://github.com/vim/vim};
	\item Sublime Text 3\\
	\href{https://www.sublimetext.com/}{https://www.sublimetext.com/};
	\item gedit;\\
\end{itemize}

\section{Sistemi operativi}
\begin{itemize}
	\item CentOS 7\\
	\href{https://www.centos.org/}{https://www.centos.org/};
	\item RHEL 7\\
	\href{https://www.redhat.com/en/store/linux-platforms}{https://www.redhat.com/en/store/linux-platforms};
	\item Microsoft Windows 7 Professional;\\
\end{itemize}
  


  
\begin{dedication}
  
\end{dedication}
\begin{dedication}
  \textit{Dedica a fine pagina}
\end{dedication}

\end{document}

%%%%%%%%%%%%%%%%%%%%%%%%%%%%%%%%%%%%%%%%%%%%%%%%%%%%%%%%%%%%%%%%%%%%%%%%%%%%%%%%

