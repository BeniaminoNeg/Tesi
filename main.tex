\documentclass[hidelinks,a4paper,12pt]{tesiinfo}

%%Pacchetti utili anche se non necessari
\usepackage{amsfonts}

\usepackage{amsmath}

\usepackage{latexsym}

\usepackage{tabularx}
%% Supporto alla lingua italiana, dipende dal vostro ambiente latex
\usepackage[italian]{babel}
%% Supporto per i caratteri accentati
\usepackage[utf8]{inputenc}
%% Supporto alle figure
\usepackage{graphicx}
%% Supporto alle figure wrappate nel testo
\usepackage{wrapfig}
%% Supporto alle didascalie
\usepackage{caption}
%% Supporto alle figure composte
\usepackage{subfigure}
%% Supporto alle didascalie per figure composte
%%\usepackage{subcaption}
%% Supporto ai link esterni (URL)
\usepackage[bookmarks=true]{hyperref}

%% Supporto alla creazione di colonne multiple nel documento
\usepackage{multicol}
%% Supporto al comando \cite per BibTex
\usepackage{cite}
%% Supporto per i listati di codice
\usepackage{listings}
%% Supporto allo stile per il codice PHP creato da Nicola Sacco e Daniele Di Pompeo
\usepackage{listings,xcolor}
\usepackage{caption}

\definecolor{mygreen}{rgb}{0,0.6,0}
\definecolor{mygray}{rgb}{0.5,0.5,0.5}
\definecolor{mymauve}{rgb}{0.58,0,0.82}

\lstset{ %
  backgroundcolor=\color{white},   % choose the background color; you must add \usepackage{color} or \usepackage{xcolor}
  basicstyle= \scriptsize\ttfamily,% the size of the fonts that are used for the code
  breakatwhitespace=false,         % sets if automatic breaks should only happen at whitespace
  breaklines=true,                 % sets automatic line breaking
  commentstyle=\color{mygreen},    % comment style
  deletekeywords={...},            % if you want to delete keywords from the given language
  escapeinside={\%*}{*)},          % if you want to add LaTeX within your code
  extendedchars=true,              % lets you use non-ASCII characters; for 8-bits encodings only, does not work with UTF-8
%   frame=single,                    % adds a frame around the code
  keywordstyle=\color{red},       % keyword style
  language=php,                 % the language of the code
  morekeywords={*,new},            % if you want to add more keywords to the set
%   numbers=left,                    % where to put the line-numbers; possible values are (none, left, right)
%   numbersep=5pt,                   % how far the line-numbers are from the code
%   numberstyle=\tiny\color{mygray}, % the style that is used for the line-numbers
  rulecolor=\color{black},         % if not set, the frame-color may be changed on line-breaks within not-black text (e.g. comments (green here))
  showspaces=false,                % show spaces everywhere adding particular underscores; it overrides 'showstringspaces'
  showstringspaces=false,          % underline spaces within strings only
  showtabs=false,                  % show tabs within strings adding particular underscores
  stepnumber=2,                    % the step between two line-numbers. If it's 1, each line will be numbered
  stringstyle=\color{blue},     % string literal style
  tabsize=2,                       % sets default tabsize to 2 spaces
}

\DeclareCaptionFont{white}{\color{white}\tiny}
\DeclareCaptionFormat{listing}{\colorbox[cmyk]{0.43, 0.35, 0.35,0.01}{\parbox{\textwidth}{\hspace{15pt}#1#2#3}}}
\captionsetup[lstlisting]{format=listing,labelfont=white,textfont=white, singlelinecheck=false, margin=0pt, font={bf,footnotesize}}
%% Supporto al glossario
\usepackage[style=altlist, toc=true]{glossary} % can be obtained from http://www.ctan.org/tex-archive/macros/latex/contrib/glossary/
%% Richiesta di creazione del glossario
\makeglossary
%% Inclusione del file con i termini del glossario
\storeglosentry{Esempio}{name={Nome da visualizzare}, description={Descrizione nel glossario}}



\titolo{Titolo della tesi}
\laureando{Nome del laureando}
\relatore{Nome del relatore}
\annoaccademico{Anno accademico}

%%Usare i seguenti comandi se si ha un correlatore:
%\setcorrelatoreuno
%\correlatoreuno{Nome e cognome del correlatore}

%%Usare i seguenti comandi se si hanno due correlatori (NB: questi comandi sono alternativi a quelli precendenti):
%\setcorrelatoredue
%\correlatoreuno{Nome e cognome del primo correlatore}
%\correlatoredue{Nome e cognome del secondo correlatore}

%%Usare i seguenti comandi se si ha un relatore esterno (NB: questi comandi possono essere utilizzati con quelli precedenti):
%\setesterno
%\relatoreesterno{Titolo, nome e cognome del relatore esterno}

%%Usare i seguenti comandi se si sta scrivendo una tesi di laurea specialistica
%\setspecialistica

\begin{document}


\maketitle
   
  \begin{dedication}
  \textit{Dedica a piè pagina}
  \end{dedication}



\contentspage
  
  \chapter{Introduzione}

\paragraph{}
(prima bozza, verrà raffinata al termine della stesura dei capitoli)

\paragraph{}
E' sempre più evidente che il cloud computing è il futuro del software. La rivoluzione consta nella distribuzione dei servizi di calcolo e nella virtualizzazione delle risorse, dando così all'utente la sensazione di un utilizzo centralizzato. Tutto ciò si è reso realizzabile dal momento in cui l'accesso alla rete è divenuto possibile da sempre più dispositivi e con velocità di connessione sempre maggiore.
\paragraph{}
La tematica del cloud computing è stata centrale nel mio lavoro di tesi presso l'azienda IBM (International Business Machines Corporation) nella sua sede di Roma. Ho partecipato attivamente alla realizzazione di un prototipo software, ossia la versione SaaS della suite IBM BigFix. BigFix è una suite di prodotti dedicati alle aziende che risolvono problematiche di Endpoint Security e di compliance di dispositivi a determinate politiche aziendali. Tramite questi prodotti si ottiene pieno controllo su tutti i dispositivi aziendali. Si possono ad esempio rilevare eventuali attacchi o si possono distribuire aggiornamenti e patch.
\paragraph{}
La sfida da me raccolta è quindi proprio quella di portare tutto questo arsenale di strumenti nella leggerezza del cloud. Rendendolo disponibile, nel giro di pochi minuti, anche a chi è sempre stato intimorito dalla difficoltà di installazione di uno strumento così potente, ma allo stesso tempo complesso.

\bibliografia{bibliografia/bibliografia}{}

\appendice
  \printglossary
  \chapter{Tecnologie Utilizzate(template di prova - ANCORA DA SCRIVERE)}
  \section{Linguaggi di programmazione}
    \begin{itemize}
     \item PHP 5.4.7 \\
     \href{http://www.php.net/}{http://www.php.net/};
     \item Javascript \\
     \href{http://www.w3.org/standards/webdesign/script}{http://www.w3.org/standards/webdesign/script};
    \end{itemize}
   \section{Linguaggi di Markup e Stile}
    \begin{itemize}
     \item HTML4/HTML5;
     \item CSS/CSS3;
    \end{itemize}
   \section{Framework}
    \begin{itemize}
     \item Smarty Template Engine \\
	\href{http://www.smarty.net/}{http://www.smarty.net/};
     \item JQuery\\
	\href{http://jquery.com/}{http://jquery.com/};
     \item JQueryUI\\
	\href{http://jqueryui.com/}{http://jqueryui.com/};
     \item beContent\\
	\href{http://www.becontent.org/}{http://www.becontent.org/};
    \end{itemize}
   \section{Ambiente di Sviluppo}
    \subsection{Eclipse}
      Per Eclipse sono state utilizzate due versioni differenti, la 4.2.2 in ambiente Windows e la 3.8.0 in ambiente Ubuntu/Linux \\
      \href{http://www.eclipse.org/}{http://www.eclipse.org/} \\
      Inoltre è stato utilizzato il pacchetto
      \begin{itemize}
       \item PHP Development Tools 3.1.1 \\
       \href{http://projects.eclipse.org/projects/tools.pdt}{http://projects.eclipse.org/projects/tools.pdt};
       
      \end{itemize}
    \subsection{Piattaforma Web}
      \subsubsection{XAMPP}
      \href{http://www.apachefriends.org}{http://www.apachefriends.org}
      \begin{itemize}
       \item Apache Web Server ver. 2.4.3 \\
	  \href{http://httpd.apache.org/}{http://httpd.apache.org/};
       \item MySql Database Management System ver. 5.5.27 \\
	  \href{http://dev.mysql.com/}{http://dev.mysql.com/};
      \end{itemize}
    \subsection{Browser Testing}
      \subsubsection{Mozilla Firefox}
	\begin{itemize}
	 \item Firebug ver 1.11.2 \\
	  \href{http://getfirebug.com/}{http://getfirebug.com/}
	    \begin{itemize}
	     \item Plug-In Validator ver. 0.0.6 \\
	      \href{https://addons.mozilla.org/it/firefox/addon/validator/}{https://addons.mozilla.org/it/firefox/addon/validator/};
	     \item Plug-In Google Page Speed ver. 2.0.2.3 \\
	      \href{https://developers.google.com/speed/pagespeed/?hl=it-IT}{https://developers.google.com/speed/pagespeed/?hl=it-IT};
	    \end{itemize}
	\end{itemize}
      \subsubsection{Google Chrome}
	\begin{itemize}
	 \item Strumenti per gli sviluppatori integrati 
	\end{itemize}
      \subsubsection{Responsive Testing}
	\begin{itemize}
	 \item Viewport Resizer- Responsive Design Bookmarklet \\
	 \href{http://lab.maltewassermann.com/viewport-resizer/}{http://lab.maltewassermann.com/viewport-resizer/} ;
	\end{itemize}




  


  
\begin{dedication}
  
\end{dedication}
\begin{dedication}
  \textit{Dedica a fine pagina}
\end{dedication}

\end{document}

%%%%%%%%%%%%%%%%%%%%%%%%%%%%%%%%%%%%%%%%%%%%%%%%%%%%%%%%%%%%%%%%%%%%%%%%%%%%%%%%

